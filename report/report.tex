% Preamble
% =============================================================================

% Class of the document.
\documentclass[12pt,a4paper]{article}
% article : short article.
% report  : mid-length report.
% book    : book or thesis redaction.

% Paragraph skip length (default to 0).
\setlength{\parskip}{1ex}

% Packages
% =============================================================================

% Encoding
% -----------------------------------------------------------------------------

% Babel.
\usepackage[french]{babel}
% FontEnc.
\usepackage[T1]{fontenc}
% InputEnc.
\usepackage[utf8]{inputenc}

% Text
% -----------------------------------------------------------------------------

% Acronym.
\usepackage{acronym}
% CsQuote.
\usepackage[style=french,french=guillemets]{csquotes}
% Enumerate.
\usepackage{enumerate}
% HyperRef.
\usepackage[hyperfootnotes=false,hidelinks]{hyperref}
% URL.
\usepackage{url}

% Algorithms
% -----------------------------------------------------------------------------

% Algorithm2E.
\usepackage[french,onelanguage,linesnumbered,ruled,vlined,commentsnumbered]{algorithm2e}

% Source code
% -----------------------------------------------------------------------------

% Listings.
\usepackage{listings}
% Minted.
\usepackage{minted}

% Figures
% -----------------------------------------------------------------------------

% GraphicX.
\usepackage{graphicx}
% SVG.
\usepackage{svg}
% WrapFig.
\usepackage{wrapfig}

% Charts
% -----------------------------------------------------------------------------

% PGFPLots
\usepackage{pgfplots}
\pgfplotsset{compat=1.16}
\usepgfplotslibrary{units}

% Mathematics
% -----------------------------------------------------------------------------

% AmsFonts.
\usepackage{amsfonts}
% AmsMath.
\usepackage{amsmath}
% AmsText.
\usepackage{amstext}
% AmsThm.
\usepackage{amsthm}
\newtheorem{prr}{Propriété}
\newtheorem{pro}{Proposition}
\newtheorem{thm}{Théorème}
\newtheorem{lem}{Lemme}
% NumPrint.
\usepackage{numprint}

% Physics
% -----------------------------------------------------------------------------

% Physics.
\usepackage{physics}

% Presentation
% -----------------------------------------------------------------------------

% XColor.
\usepackage{xcolor}

% References
% -----------------------------------------------------------------------------

% CleveRef.
\usepackage{cleveref}

% Structure.
% -----------------------------------------------------------------------------

% Geometry.
\usepackage{geometry}
% PDFLScape.
\usepackage{pdflscape}
% MultiCol.
\usepackage{multicol}
% TitleSec.
\usepackage{titlesec}
\newcommand{\sectionbreak}{\clearpage} % Use a page break before new sections.
% VMargin.
\usepackage{vmargin}
% FootMisc.
\usepackage[bottom]{footmisc}

% Symbols
% -----------------------------------------------------------------------------

% SIUnitX.
\usepackage{siunitx}

% Table
% -----------------------------------------------------------------------------

% Array.
\usepackage{array}
% BookTabs.
\usepackage{booktabs}
% CSVSimple.
\usepackage{csvsimple}

% Document
% =============================================================================

\begin{document}

\title{Base de données avec Oracle DBA sur les championnats de moto}
\author{Pierre AYOUB et Maël ROUXEL}

\maketitle

\begin{figure}[b]
    \centering
    \includegraphics[scale=0.3]{figures/isty.jpg}
\end{figure}

\newpage
\begin{abstract}
    
Oracle Database est un système de gestion de bases de données relationnelles
(SGBD) utilisé dans le monde entier. Très répandu en entreprise, tant pour
ses performances que sa fiabilité, nous utilisons ce SGBD afin de créer une
base de données et d’y effectuer des tâches d’administration. Plusieurs
possibilités offertes par Oracle DBA seront explorées dans ce projet.
    
\end{abstract}

\tableofcontents

\section{Introduction}
\label{sec.intro}

Notre projet modélise une base de données concernant les championnats de moto.
L’objectif de cette base est de stocker des informations non pas sur une seule
saison de course, mais sur plusieurs saisons. De plus, on pourra stocker au sein
d’une même base plusieurs championnats différents. Beaucoup d'informations
techniques sont retenues concernant les motos et les résultats des pilotes sur
chaques courses, ce qui permettra d’obtenir des statistiques poussées et
diversifiées. La base de données possède quelques contraintes, listées
ci-dessous (pour les moins évidentes) :

\begin{itemize}
    \item Une saison d'un championnat dure une année.

    \item Par saison, un pilote peut participer à plusieurs championnats.

    \item Pour un championnat donné, un pilote ne peut faire partit que d'une
        team. Dans le cas où le pilote participe à plus d'un championnat sur une
        saison, alors il peut faire partit de plusieurs teams différentes
        concourantes sur différents championnats.

    \item Pour un championnat donné et une team donné, un pilote ne peut
        utiliser qu'une moto. Dans le cas où le pilote participe à plus d'un
        championnat différents sur une saison, alors il peut utiliser plusieurs
        motos différentes sur les différents championnats.

    \item Chaque pilote doit être sous contrat pour pouvoir courir dans un
        championnat. Un contrat est  un CDD liant un pilote, un modèle de moto
        et une team pendant un temps donné (généralement, quelques années).

    \item La participation d’un pilote à un course correspond à une relation
        entre ladite course et le contrat d’un pilote.
\end{itemize}

\begin{listing}
    \begin{minted}[linenos,numbersep=5pt,frame=lines,framesep=2mm]{SQL}
    TODO
    \end{minted}
    \caption{Code SQL}
    \label{lst.sql}
\end{listing}

\section{Conclusion}
\label{sec.conc}

Ce projet nous aura beaucoup appris concernant Oracle DBA et PL/SQL. Concernant
Oracle DBA, nous aurons constaté qu’il existe un large écosystème d’outil de
développement autour de ce SGBD, des outils de débogage jusqu’à l’analyse de
performance en passant par des utilitaires facilitant la manipulation des
données. Par rapport à PL/SQL, nous avons pu expérimenter différentes
utilisations du langage, par exemple pour gérer des contraintes avancées ou
encore permettre d’automatiser certaines opérations de gestion des données
nécessitant un programme dynamique. Pour conclure, ce projet aura été une bonne
approche et une introduction intéressante à l’administration de base de données.

\end{document}
