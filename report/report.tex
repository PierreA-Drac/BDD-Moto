% Preamble
% =============================================================================

% Class of the document.
\documentclass[12pt,a4paper]{article}
% article : short article.
% report  : mid-length report.
% book    : book or thesis redaction.

% Paragraph skip length (default to 0).
\setlength{\parskip}{1ex}

% Packages
% =============================================================================

% Encoding
% -----------------------------------------------------------------------------

% Babel.
\usepackage[french]{babel}
% FontEnc.
\usepackage[T1]{fontenc}
% InputEnc.
\usepackage[utf8]{inputenc}

% Text
% -----------------------------------------------------------------------------

% Acronym.
\usepackage{acronym}
% CsQuote.
\usepackage[style=french,french=guillemets]{csquotes}
% Enumerate.
\usepackage{enumerate}
% HyperRef.
\usepackage[hyperfootnotes=false,hidelinks]{hyperref}
% URL.
\usepackage{url}

% Algorithms
% -----------------------------------------------------------------------------

% Algorithm2E.
\usepackage[french,onelanguage,linesnumbered,ruled,vlined,commentsnumbered]{algorithm2e}

% Source code
% -----------------------------------------------------------------------------

% Listings.
\usepackage{listings}
% Minted.
\usepackage{minted}
% Caption.
\usepackage{caption}
\newenvironment{code}{\captionsetup{type=listing}}{}

% Figures
% -----------------------------------------------------------------------------

% GraphicX.
\usepackage{graphicx}
% SVG.
\usepackage{svg}
% WrapFig.
\usepackage{wrapfig}

% Charts
% -----------------------------------------------------------------------------

% PGFPLots
\usepackage{pgfplots}
\pgfplotsset{compat=1.16}
\usepgfplotslibrary{units}

% Mathematics
% -----------------------------------------------------------------------------

% AmsFonts.
\usepackage{amsfonts}
% AmsMath.
\usepackage{amsmath}
% AmsText.
\usepackage{amstext}
% AmsThm.
\usepackage{amsthm}
\newtheorem{prr}{Propriété}
\newtheorem{pro}{Proposition}
\newtheorem{thm}{Théorème}
\newtheorem{lem}{Lemme}
% NumPrint.
\usepackage{numprint}

% Physics
% -----------------------------------------------------------------------------

% Physics.
\usepackage{physics}

% Presentation
% -----------------------------------------------------------------------------

% XColor.
\usepackage{xcolor}

% References
% -----------------------------------------------------------------------------

% CleveRef.
\usepackage{cleveref}

% Structure.
% -----------------------------------------------------------------------------

% Geometry.
\usepackage{geometry}
% PDFLScape.
\usepackage{pdflscape}
% MultiCol.
\usepackage{multicol}
% TitleSec.
\usepackage{titlesec}
\newcommand{\sectionbreak}{\clearpage} % Use a page break before new sections.
% VMargin.
\usepackage{vmargin}
% FootMisc.
\usepackage[bottom]{footmisc}

% Symbols
% -----------------------------------------------------------------------------

% SIUnitX.
\usepackage{siunitx}

% Table
% -----------------------------------------------------------------------------

% Array.
\usepackage{array}
% BookTabs.
\usepackage{booktabs}
% CSVSimple.
\usepackage{csvsimple}

% Document
% =============================================================================

\begin{document}

\title{Base de données avec Oracle DBA sur les championnats de moto}
\author{Pierre AYOUB et Maël ROUXEL}

\maketitle

\begin{figure}[b]
    \centering
    \includegraphics[scale=0.3]{figures/isty.jpg}
\end{figure}

\newpage
\begin{abstract}
    
Oracle Database est un système de gestion de bases de données relationnelles
(SGBD) utilisé dans le monde entier. Très répandu en entreprise, tant pour
ses performances que sa fiabilité, nous utilisons ce SGBD afin de créer une
base de données et d’y effectuer des tâches d’administration. Plusieurs
possibilités offertes par Oracle DBA seront explorées dans ce projet.
    
\end{abstract}

\tableofcontents

\section{Introduction}
\label{sec.intro}

Notre projet modélise une base de données concernant les championnats de moto.
L’objectif de cette base est de stocker des informations non pas sur une seule
saison de course, mais sur plusieurs saisons. De plus, on pourra stocker au sein
d’une même base plusieurs championnats différents. Beaucoup d'informations
techniques sont retenues concernant les motos et les résultats des pilotes sur
chaques courses, ce qui permettra d’obtenir des statistiques poussées et
diversifiées. La base de données possède quelques contraintes, listées
ci-dessous (pour les moins évidentes) :

\begin{itemize}
    \item Une saison d'un championnat dure une année.

    \item Par saison, un pilote peut participer à plusieurs championnats.

    \item Pour un championnat donné, un pilote ne peut faire partit que d'une
        team. Dans le cas où le pilote participe à plus d'un championnat sur une
        saison, alors il peut faire partit de plusieurs teams différentes
        concourantes sur différents championnats.

    \item Pour un championnat donné et une team donné, un pilote ne peut
        utiliser qu'une moto. Dans le cas où le pilote participe à plus d'un
        championnat différents sur une saison, alors il peut utiliser plusieurs
        motos différentes sur les différents championnats.

    \item Chaque pilote doit être sous contrat pour pouvoir courir dans un
        championnat. Un contrat est  un CDD liant un pilote, un modèle de moto
        et une team pendant un temps donné (généralement, quelques années).

    \item La participation d’un pilote à un course correspond à une relation
        entre ladite course et le contrat d’un pilote.
\end{itemize}

\section{Mise en place de la base de données}
\label{sec.reference}

Dans cette section, nous vous présenterons la mise en place de la base de
données. Pour la majorité du travail ci-dessous, cela concerne le langage de
requête SQL ou l’utilisation de l’utilitaire de chargement de données Oracle SQL
Loader.

\subsection{Schéma de la base de données}
\label{sub.scheme}

La création du schéma de la base de données consiste à créer les tables en
spécifiant les attributs, leurs types, et leurs contraintes d'intégrités
basiques tel que les clés primaires et étrangères, ainsi que les contraintes
check.

\begin{code}
    \begin{minted}[linenos,numbersep=5pt,frame=lines,framesep=2mm,breaklines=true]{SQL}
    -- Création des tables.

    -- 1. Marques de moto.
    CREATE TABLE Marque
    (
        Nom         VARCHAR(32) NOT NULL,
        Annee       DATE,       -- CHECK with a trigger.
        Nationalite CHAR(2),    -- CHECK with a trigger.
        PRIMARY KEY (Nom)
    );

    -- 2. Teams concourrant aux championnats.
    CREATE TABLE Team
    (
        Nom    VARCHAR(32) NOT NULL,
        Marque VARCHAR(32) NOT NULL,
        PRIMARY KEY (nom)
    );

    -- 3. Modèles de moto.
    CREATE TABLE Modele_moto
    (
        Marque      VARCHAR(32) NOT NULL,
        Nom         VARCHAR(32) NOT NULL,
        Annee       DATE        NOT NULL, -- CHECK with a trigger.
        Cylindree   FLOAT       CHECK (Cylindree > 20 AND Cylindree < 2000),
        Couple      FLOAT       CHECK (Couple > 1     AND Couple < 20),
        Puissance   FLOAT       CHECK (Puissance > 1  AND Puissance < 500),
        Poids       FLOAT       CHECK (Poids > 30     AND Poids < 500),
        Prix        NUMBER(6)   CHECK (Prix > 100 AND Prix < 500000),
        Genre       VARCHAR(40) NOT NULL CHECK (Genre IN ('Sportive', 'Cafe Racer')),
        PRIMARY KEY (nom, annee)
    );

    -- 4. Pilotes appartenant aux teams.
    CREATE TABLE Pilote
    (
        Id          NUMBER(4)   NOT NULL, -- CHECK with a trigger.
        Nom         VARCHAR(32) NOT NULL,
        Prenom      VARCHAR(32) NOT NULL,
        Age         NUMBER(3)   CHECK (Age BETWEEN 10 and 100),
        Nationalite CHAR(2),    -- CHECK with a trigger.
        Sexe        CHAR(1)     CHECK (Sexe IN (NULL, 'H', 'F')),
        Numero      NUMBER(2)   CHECK (Numero BETWEEN 0 and 99),
        PRIMARY KEY (id)
    );

    -- 5. Championnats existants.
    CREATE TABLE Championnat
    (
        Nom         VARCHAR(32) NOT NULL,
        Annee       DATE        NOT NULL,   -- CHECK with a trigger.
        PRIMARY KEY (Nom, Annee)
    );

    -- 6. Circuits sur lesquels les courses se déroulent.
    CREATE TABLE Circuit
    (
        Nom      VARCHAR(32) NOT NULL,
        Pays     CHAR(2)     NOT NULL,  -- CHECK with a trigger.
        Longueur FLOAT       CHECK (Longueur BETWEEN 0.5 AND 20),
        PRIMARY KEY (Nom)
    );

    -- 7. Courses appartenants aux championnats.
    CREATE TABLE Course_vitesse
    (
        Championnat VARCHAR(32) NOT NULL,
        Annee       DATE        NOT NULL,
        Date_course DATE        NOT NULL, -- CHECK with trigger.
        Circuit     VARCHAR(32) NOT NULL,
        Nb_tours    NUMBER(2)   CHECK (Nb_tours BETWEEN 1 AND 30),
        Duree       FLOAT       CHECK (Duree BETWEEN 1 AND 100),
        PRIMARY KEY (Championnat, Date_course)
    );

    -- 8. Participation d'un pilote à une course.
    CREATE TABLE Participe (
        Id_pilote       INT         NOT NULL,
        Championnat     VARCHAR(32) NOT NULL,
        Date_course     DATE        NOT NULL,
        Modele_moto     VARCHAR(32) NOT NULL,
        Annee_moto      DATE        NOT NULL,
        Classement      NUMBER(2)   CHECK (Classement BETWEEN 0 AND 30),
        Points_gagnes   NUMBER(2)   CHECK (Points_gagnes BETWEEN 0 AND 25),
        Vitesse_moy     FLOAT       CHECK (Vitesse_moy BETWEEN 0 AND 300),
        Meilleur_tour   FLOAT       CHECK (Meilleur_tour BETWEEN 0 AND 400),
        PRIMARY KEY (Id_pilote, Championnat, Date_course, Modele_moto, Annee_moto)
    );

    -- 9. Contrats liants un pilote, une équipe et un modèle de moto.
    CREATE TABLE Contrat (
        Id_pilote   NUMBER(4)   NOT NULL,
        Moto_modele VARCHAR(32) NOT NULL,
        Moto_annee  DATE        NOT NULL,
        Team_nom    VARCHAR(32) NOT NULL,
        Annee_debut DATE        NOT NULL,
        Annee_fin   DATE        NOT NULL,
        PRIMARY KEY (Id_pilote, Moto_modele, Moto_annee, Team_nom, Annee_debut)
    );

    -- Configure les clés étrangères.

    ALTER TABLE Team
        ADD FOREIGN KEY (Marque) REFERENCES Marque (Nom);

    ALTER TABLE Modele_moto
        ADD FOREIGN KEY (Marque) REFERENCES Marque (Nom);
        
    -- Désactivé car on utilise un trigger pour ces clés, conformément à la consigne.
    --  ALTER TABLE Course_vitesse
        --  ADD FOREIGN KEY (Championnat, Annee) REFERENCES Championnat (Nom, Annee);
    ALTER TABLE Course_vitesse
        ADD FOREIGN KEY (Circuit) REFERENCES Circuit (Nom);

    ALTER TABLE Participe
        ADD FOREIGN KEY (Id_pilote) REFERENCES Pilote (Id);
    ALTER TABLE Participe
        ADD FOREIGN KEY (Championnat, Date_course) REFERENCES Course_vitesse (Championnat, Date_course);
    ALTER TABLE Participe
        ADD FOREIGN KEY (Modele_moto, Annee_moto) REFERENCES Modele_moto (Nom, Annee);

    ALTER TABLE Contrat
        ADD FOREIGN KEY (Id_pilote) REFERENCES Pilote (Id);
    ALTER TABLE Contrat
        ADD FOREIGN KEY (Moto_modele, Moto_annee) REFERENCES Modele_moto (Nom, Annee);
    ALTER TABLE Contrat
        ADD FOREIGN KEY (Team_nom) REFERENCES Team (Nom);
    \end{minted}
    \caption{Code SQL permettant de mettre en place la base de données}
    \label{lst.create}
\end{code}

\subsection{Jeu de données}
\label{sub.data}

Pour le chargement du jeu de données, nous avons utilisé l'utilitaire spécialisé
Oracle SQL Loader. Du fait que cela ne serait pas pertinent d’inclure
l’intégralité du jeu de données dans le rapport, nous allons uniquement vous
présenter un exemple d’un fichier de données CSV et d’un fichier de contrôle
CTL. Le fichier CSV contient les valeurs des données allant s’intégrer dans les
tables crées précédemment, tandis que le fichier CTL contient des informations
sur la manière dont les données doivent être chargées depuis le fichier CSV. Par
exemple, des précisions sur le type de données, tel que le format de la date.

\begin{code}
    \begin{minted}[linenos,numbersep=5pt,frame=lines,framesep=2mm,breaklines=true]{SQL}
    LOAD DATA
    INFILE './Data/Participe.csv'
    TRUNCATE
    INTO TABLE Participe
    FIELDS TERMINATED BY ';'
    TRAILING NULLCOLS
    (
        Id_pilote,
        Championnat,
        Date_course DATE "YYYY-MM-DD",
        Modele_moto,
        Annee_moto DATE "YYYY",
        Classement,
        Points_gagnes,
        Vitesse_moy,
        Meilleur_tour
    )
    \end{minted}
    \caption{Code SQL Loader permettant de charger des données dans une table}
    \label{lst.loadctl}
\end{code}

\begin{code}
    \begin{minted}[linenos,numbersep=5pt,frame=lines,framesep=2mm,breaklines=true]{SQL}
    Id_pilote;Championnat;Date_course;Modele_moto;Annee_moto; Classement;Points_gagnes;Vitesse_moy;Meilleur_tour
    1;MotoGP;2016-03-20;M1;2016;1;25;167.1;114.543
    0;MotoGP;2016-03-20;Desmosedici GP;2013;2;20;167.0;;
    7;MotoGP;2016-03-20;RC213V;2012;3;16;167.0;;
    4;MotoGP;2016-03-20;M1;2016;4;13;167.0;;
    8;MotoGP;2016-03-20;RC213V;2012;5;11;166.2;;
    5;MotoGP;2016-03-20;GSX-RR;2014;6;10;166.1;;
    11;MotoGP;2016-03-20;RC213V-RS;2015;14;2;164.4;;
    6;MotoGP;2016-03-20;RC213V;2012;0;0;165.0;;
    10;MotoGP;2016-03-20;GSX-RR;2014;18;0;;;
    4;MotoGP;2016-04-24;M1;2016;1;25;157.5;100.090
    [...]
    13;Superbike;2015-02-22;GSX-R1000;2014;9;7;;;92.690
    14;Superbike;2015-02-22;ZX-10R;2015;6;10;;;92.016
    \end{minted}
    \caption{Fichier CSV contenant des données à charger (extrait)}
    \label{lst.loadcsv}
\end{code}

\subsection{Manipulation des données par requêtes SQL}
\label{sub.req}

TODO Maël

\subsection{Vues}
\label{sub.views}

Notre base de données contient quelques vues permettant de visualiser des scores
et des statistiques calculés à partir de notre jeu de données. De tels vues se
destinerait à être inclus dans un site ou service web permettant de consulter
des statistiques sur les championnats de moto, avec des paramètres dynamiques
tel que, par exemple, l'année des scores pour ledit championnat.

\begin{code}
    \begin{minted}[linenos,numbersep=5pt,frame=lines,framesep=2mm,breaklines=true]{SQL}
    -- Création des vues.

    -- 1. Liste des scores des pilotes au MotoGP de 2016.
    CREATE VIEW MotoGP_2016_Score_pilotes AS
        SELECT Pi.Id, Pi.Numero, Pi.Nom, Pi.Prenom, SUM(Pa.Points_gagnes) AS Nombre_total_de_point
        FROM Participe Pa, Pilote Pi
        WHERE Pa.Id_pilote = Pi.Id
            AND Pa.Championnat LIKE 'MotoGP'
            AND TO_CHAR(Pa.Date_course, 'YYYY') LIKE '2016'
        GROUP BY Pi.Id, Pi.Numero, Pi.Nom, Pi.Prenom
        ORDER BY Nombre_total_de_point DESC
        WITH READ ONLY;
    GRANT SELECT ON MotoGP_2016_Score_pilotes to PUBLIC;
        
    -- 2. Liste des scores des teams au MotoGP de 2016.
    CREATE VIEW MotoGP_2016_Score_teams AS
        SELECT C.Team_nom, SUM(Nombre_total_de_point) AS Nombre_total_de_point
        FROM Contrat C, MotoGP_2016_Score_pilotes S
        WHERE S.Id = C.Id_pilote
            AND TO_DATE(2016, 'YYYY') BETWEEN C.Annee_debut AND C.Annee_fin
        GROUP BY C.Team_nom
        ORDER BY Nombre_total_de_point DESC
        WITH READ ONLY;
    GRANT SELECT ON MotoGP_2016_Score_teams to PUBLIC;

    -- 3. Liste des scores des constructeurs au MotoGP de 2016.
    CREATE VIEW MotoGP_2016_Score_construc AS
        SELECT T.Marque, SUM(Nombre_total_de_point) AS Nombre_total_de_point
        FROM Team T, MotoGP_2016_Score_teams S
        WHERE T.Nom = S.Team_nom
        GROUP BY T.Marque
        ORDER BY Nombre_total_de_point DESC
        WITH READ ONLY;
    GRANT SELECT ON MotoGP_2016_Score_construc to PUBLIC;

    -- 4. Statistiques diverses sur les pilotes du MotoGP.
    CREATE VIEW MotoGP_Pilote_stat AS
        SELECT Pi.Numero, Pi.Nom, Pi.Prenom, Pi.Age, Pi.Nationalite, Pi.Sexe,
            SUM(Pa.Points_gagnes) AS Total_de_points_gagnes,
            MIN(Pa.Classement)    AS Meilleur_classement,
            MAX(Pa.Vitesse_moy)   AS Vitesse_moyenne,
            MIN(Pa.Meilleur_tour) AS Meilleur_tour
        FROM Pilote Pi, Participe Pa
        WHERE Pa.Id_pilote = Pi.Id
            AND Pa.Championnat LIKE 'MotoGP'
        GROUP BY Pi.Numero, Pi.Nom, Pi.Prenom, Pi.Age, Pi.Nationalite, Pi.Sexe
        WITH READ ONLY;
    GRANT SELECT ON MotoGP_Pilote_stat to PUBLIC;

    -- 5. Nombre de victoire des pilotes au MotoGP.
    CREATE VIEW MotoGP_Pilote_win AS
        SELECT Pi.Numero, Pi.Nom, Pi.Prenom, COUNT(*) AS Nombre_de_victoire
        FROM Pilote Pi, Participe Pa
        WHERE Pa.Id_pilote = Pi.Id
            AND Pa.Championnat LIKE 'MotoGP'
            AND Pa.Classement = 1
        GROUP BY Pi.Numero, Pi.Nom, Pi.Prenom
        ORDER BY Nombre_de_victoire DESC
        WITH READ ONLY;
    GRANT SELECT ON MotoGP_Pilote_win to PUBLIC;
    \end{minted}
    \caption{Code SQL permettant de créer les vues de la base de données}
    \label{lst.create}
\end{code}

\section{SQL Avancé}
\label{sec.reference}

\subsection{Triggers}
\label{sub.trig}

\subsection{Méta-données}
\label{sub.metadata}

\subsection{Analyse des requêtes}
\label{sub.analyze}

\section{Conclusion}
\label{sec.conc}

Ce projet nous aura beaucoup appris concernant Oracle DBA et PL/SQL. Concernant
Oracle DBA, nous aurons constaté qu’il existe un large écosystème d’outil de
développement autour de ce SGBD, des outils de débogage jusqu’à l’analyse de
performance en passant par des utilitaires facilitant la manipulation des
données. Par rapport à PL/SQL, nous avons pu expérimenter différentes
utilisations du langage, par exemple pour gérer des contraintes avancées ou
encore permettre d’automatiser certaines opérations de gestion des données
nécessitant un programme dynamique. Pour conclure, ce projet aura été une bonne
approche et une introduction intéressante à l’administration de base de données.

\end{document}
